% This file was converted to LaTeX by Writer2LaTeX ver. 1.4
% see http://writer2latex.sourceforge.net for more info
\documentclass[letterpaper]{article}
\usepackage[latin1]{inputenc}
\usepackage[T1]{fontenc}
\usepackage[english]{babel}
\usepackage{amsmath}
\usepackage{amssymb,amsfonts,textcomp}
\usepackage{color}
\usepackage{array}
\usepackage{supertabular}
\usepackage{hhline}
\usepackage{hyperref}
\hypersetup{pdftex, colorlinks=true, linkcolor=blue, citecolor=blue, filecolor=blue, urlcolor=blue, pdftitle=, pdfauthor=Jenifer Cochran, pdfsubject=, pdfkeywords=}
\usepackage[pdftex]{graphicx}
% Outline numbering
\setcounter{secnumdepth}{0}
\makeatletter
\newcommand\arraybslash{\let\\\@arraycr}
\makeatother
% List styles
\newcommand\liststyleWWNumi{%
\renewcommand\theenumi{\arabic{enumi}}
\renewcommand\theenumii{\alph{enumii}}
\renewcommand\theenumiii{\roman{enumiii}}
\renewcommand\theenumiv{\arabic{enumiv}}
\renewcommand\labelenumi{\theenumi.}
\renewcommand\labelenumii{\theenumii.}
\renewcommand\labelenumiii{\theenumiii.}
\renewcommand\labelenumiv{\theenumiv.}
}
\newcommand\liststyleWWNumii{%
\renewcommand\theenumi{\arabic{enumi}}
\renewcommand\theenumii{\alph{enumii}}
\renewcommand\theenumiii{\roman{enumiii}}
\renewcommand\theenumiv{\arabic{enumiv}}
\renewcommand\labelenumi{\theenumi.}
\renewcommand\labelenumii{\theenumii.}
\renewcommand\labelenumiii{\theenumiii.}
\renewcommand\labelenumiv{\theenumiv.}
}
% Page layout (geometry)
\setlength\voffset{-1in}
\setlength\hoffset{-1in}
\setlength\topmargin{1in}
\setlength\oddsidemargin{1in}
\setlength\textheight{9.0in}
\setlength\textwidth{6.5in}
\setlength\footskip{0.0cm}
\setlength\headheight{0cm}
\setlength\headsep{0cm}
% Footnote rule
\setlength{\skip\footins}{0.0469in}
\renewcommand\footnoterule{\vspace*{-0.0071in}\setlength\leftskip{0pt}\setlength\rightskip{0pt plus 1fil}\noindent\textcolor{black}{\rule{0.25\columnwidth}{0.0071in}}\vspace*{0.0398in}}
% Pages styles
\makeatletter
\newcommand\ps@Standard{
  \renewcommand\@oddhead{}
  \renewcommand\@evenhead{}
  \renewcommand\@oddfoot{}
  \renewcommand\@evenfoot{}
  \renewcommand\thepage{\arabic{page}}
}
\makeatother
\pagestyle{Standard}
\setlength\tabcolsep{1mm}
\renewcommand\arraystretch{1.3}
\title{}
\author{Jenifer Cochran}
\date{2017-11-27}
\begin{document}
\clearpage\setcounter{page}{1}\pagestyle{Standard}
{\raggedleft
Jenifer Cochran
\par}

{\raggedleft
CS 633
\par}

{\raggedleft
Homework 2
\par}


\bigskip

{\centering
Voronoi and Hedcut
\par}


\bigskip

\section{Understanding the implementation}

\bigskip

\ \ \ \ \ For many years newspapers and artists used the stippling technique to construct drawings. \ The stippling
technique use to be done by hand which is placing small dots of ink onto a paper where darker areas have a higher
density of dots and whiter areas have less or no dots. An important quality of stippling is that the dots are well
spaced. \ There are two algorithms that can do this technique programmatically with little user input which are Lloyd's
method (Voronoi) and weighted Voronoi diagram (Hedcut). \ 

\ \ \ \ Lloyd's method (Voronoi) has the following algorithm steps:

\liststyleWWNumi
\begin{enumerate}
\item Generate n random points inside the interested region
\item Generate the Voronoi diagram of the n points
\item Find the centroid (center of mass) of each Voronoi cell by sampling
\item Use the n centroids as the resulting points
\item If none of the centroids moved (or reached max iterations) then stop, otherwise go to step 2
\end{enumerate}
\ \ \ \ \ The most important part of the Lloyd's method is the generation and the centroid finding. \ Notice in the
following image the black dot represents the Voronoi calculated site, and the red dot is the centroid of the Voronoi
cell.


\includegraphics[width=2.4362in,height=2.328in]{VoronoiHedcutPaperWithPictures-img/VoronoiHedcutPaperWithPictures-img001.jpg}


\ \ \ \ \ The centroid is calculated by both the distance and the density of the Voronoi cell. \ The density is
calculated by chopping the image up into tiles and samples the density from a point in each tile. \ Then it uses the
average density of all the tiles within each cell. \ Until there is no difference between the black dot and the red dot
(the site and the centroid), the Lloyd's method continues the relaxation portion of the algorithm until the difference
is negligible or we reached the maximum number of iterations. Voronoi weighted algorithm (Hedcut) is very similar but
with one significant difference.

\ \ \ Voronoi weighted algorithm is the following steps:

\liststyleWWNumii
\begin{enumerate}
\item Generate n random points inside the interested region 
\item Generate the Voronoi diagram for n points
\item Compute the Centroid using all the pixels within a cell
\item Use the n weighted centroids as the resulting points
\item If none of the centroids moved (or max iterations reached) stop, otherwise go to step 2
\end{enumerate}
\ \ \ \ The computation of the centroid is what sets these two algorithms apart. \ Instead of using sampling to
calculate the density of the Voronoi cell, Hedcut computes the exact density by integrating over all pixels in the
cell. \ This will affect the quality of region clustering and be able to compute the edges more accurately than
sampling would. \ This also makes Hedcut run significantly slower than Voronoi since it is using every pixel in its
density calculation.

\ \ \ \ Another way these two algorithms are set apart are the stipple placements. \ In Voronoi the stipple placement is
always at the highest intensity in the cell whereas Hedcut places the stipple at the site location of the cell. \ This
difference makes edges more prominent in Voronoi than in Hedcut. \ \ 

\section{Compare the outputs}

\bigskip

When running the same program on the same image multiple times using Voronoi, the image is the same. \ Normally this
isn't the case but since in the code it doesn't seed the random object, the randomization is always the same. \ Thus,
leading to the same output results.

Here are two images of running Voronoi two separate times with 8000 stipples:


\includegraphics[width=0.9138in,height=1.4693in]{VoronoiHedcutPaperWithPictures-img/VoronoiHedcutPaperWithPictures-img002.png}

\includegraphics[width=0.8866in,height=1.4264in]{VoronoiHedcutPaperWithPictures-img/VoronoiHedcutPaperWithPictures-img003.png}


\ \ \ \ When running the same program on the same image multiple times using Hedcut, the image is different due to the
randomization of sample points. \ Here is an example using the same image two separate times with 8000 stipples:


\includegraphics[width=0.8909in,height=1.4335in]{VoronoiHedcutPaperWithPictures-img/VoronoiHedcutPaperWithPictures-img004.png}

\includegraphics[width=0.9256in,height=1.489in]{VoronoiHedcutPaperWithPictures-img/VoronoiHedcutPaperWithPictures-img005.png}


\ \ \ It is hard to see when the image is small but the stipples are in different locations. If you look at the right
button of Klayman, the image on the right has a darker left side portion than the left image's right-most button. There
are other variations throughout the image, all due to the randomization of sample points.

\ \ When you vary the number of disks on the output images for both Voronoi and Hedcut the distribution differs in the
final image. \ The following images are of Hedcut with stipples of 4,000; 8,000; 16,000; 32,000; 64,000; 128,000; and
256,000.


\bigskip


\includegraphics[width=1.3898in,height=1.8528in]{VoronoiHedcutPaperWithPictures-img/VoronoiHedcutPaperWithPictures-img006.png}

\includegraphics[width=1.3925in,height=1.8563in]{VoronoiHedcutPaperWithPictures-img/VoronoiHedcutPaperWithPictures-img007.png}

\includegraphics[width=1.3909in,height=1.8547in]{VoronoiHedcutPaperWithPictures-img/VoronoiHedcutPaperWithPictures-img008.png}

\includegraphics[width=1.3909in,height=1.8547in]{VoronoiHedcutPaperWithPictures-img/VoronoiHedcutPaperWithPictures-img009.png}

\includegraphics[width=1.4091in,height=1.8791in]{VoronoiHedcutPaperWithPictures-img/VoronoiHedcutPaperWithPictures-img010.png}

\includegraphics[width=1.4091in,height=1.8783in]{VoronoiHedcutPaperWithPictures-img/VoronoiHedcutPaperWithPictures-img011.png}

\includegraphics[width=1.4138in,height=1.8854in]{VoronoiHedcutPaperWithPictures-img/VoronoiHedcutPaperWithPictures-img012.png}
 \ 

\ As the stipples increase for Hedcut you can see more detail up to 64,000 stipples and when we hit 128,000 stipples we
end up seeing the darkest areas of the photo to have less stipples. \ This is because we are having collisions in the
darker areas and when sites cluster too closely together, it causes cells to have zero coverage, and therefore the cell
is removed. \ This is due to the resolution. \ If we had a higher resolution photo we wouldn't run into this problem so
quickly, but eventually the same thing will occur the more stipples we add. 


\bigskip

\ \ As for Voronoi the distribution is different with the same number of stipples (4,000; 8,000; 16,000; 32,000; 64,000;
128,000; 256,000):


\includegraphics[width=1.2898in,height=1.7193in]{VoronoiHedcutPaperWithPictures-img/VoronoiHedcutPaperWithPictures-img013.png}
 \ \ \ \ \ \ 
\includegraphics[width=1.2925in,height=1.7228in]{VoronoiHedcutPaperWithPictures-img/VoronoiHedcutPaperWithPictures-img014.png}
 \ \ \ \ \ 
\includegraphics[width=1.2925in,height=1.7228in]{VoronoiHedcutPaperWithPictures-img/VoronoiHedcutPaperWithPictures-img015.png}
 \ \ \ \ \ 
\includegraphics[width=1.298in,height=1.7299in]{VoronoiHedcutPaperWithPictures-img/VoronoiHedcutPaperWithPictures-img016.png}



\includegraphics[width=1.2925in,height=1.7228in]{VoronoiHedcutPaperWithPictures-img/VoronoiHedcutPaperWithPictures-img017.png}
 \ \ \ \ \ \ 
\includegraphics[width=1.298in,height=1.7299in]{VoronoiHedcutPaperWithPictures-img/VoronoiHedcutPaperWithPictures-img018.png}
 \ \ \ \ \ 
\includegraphics[width=1.3193in,height=1.7602in]{VoronoiHedcutPaperWithPictures-img/VoronoiHedcutPaperWithPictures-img019.png}


\ \ \ \ \ \ For Voronoi, you do not have the problem where the cell coverage is zero therefore removed. \ This algorithm
does not remove overlapping stipples, therefore you see a different distribution. 

\ \ \ When you vary the disks in the output images there is a significant performance difference between Voronoi and
Hedcut. \ Voronoi is much faster than Hedcut as the number of disks increase. ������As the pixels increase Hedcut is
much slower since it uses every pixel to calculate the density of a Voronoi cell. \ Though there is a point where if
there are too few stipples then the centroids move a around a great distance which can cause some more iterations. \ In
general, as the number of disks increase, Voronoi is significantly faster than Hedcut.

\ \ \ Another variance that impacts the algorithms is the number of pixels in the image. The number of pixels effects
Hedcut greatly. \ As you noticed with the ``Fairyeyes.png'' photo, if there are not enough pixels, you will end up with
collision of sites where cells will be removed (since the cell will have zero coverage). \ As for Voronoi, the number
of pixels also effects its performance but will not experience the remove of cells problem. \ 

\ \ Brightness is also another factor that can alter the outputs. \ The brightness will affect the way Voronoi places
the site, since the stipples are placed at the highest intensity pixel within the cell. \ Hedcut uses the center of the
site therefore will not be effected as much as far as stipple placement. \ \ 

\ \ \ \ I ran the algorithms with various types of images, natural versus urban landscapes as well as photos versus
paintings. \ 

Here is Voronoi with 40,000 stipples for Urban:


\includegraphics[width=2.7484in,height=1.5457in]{VoronoiHedcutPaperWithPictures-img/VoronoiHedcutPaperWithPictures-img020.png}

\includegraphics[width=2.7398in,height=1.5409in]{VoronoiHedcutPaperWithPictures-img/VoronoiHedcutPaperWithPictures-img021.png}


Here is Hedcut with 40,000 stipples for Urban:


\includegraphics[width=2.7484in,height=1.5457in]{VoronoiHedcutPaperWithPictures-img/VoronoiHedcutPaperWithPictures-img020.png}

\includegraphics[width=2.7835in,height=1.5646in]{VoronoiHedcutPaperWithPictures-img/VoronoiHedcutPaperWithPictures-img022.png}



\bigskip

\ \ In this example I think that Voronoi does better for the Urban landscape. \ The edges are much sharper than the one
in Hedcut.

Here is Voronoi with 40,000 stipples for Landscape:


\includegraphics[width=1.9598in,height=0.6134in]{VoronoiHedcutPaperWithPictures-img/VoronoiHedcutPaperWithPictures-img023.jpg}

\includegraphics[width=1.9335in,height=0.6047in]{VoronoiHedcutPaperWithPictures-img/VoronoiHedcutPaperWithPictures-img024.png}


Here is Hedcut with 40,000 stipples for Landscape:


\includegraphics[width=1.9598in,height=0.6134in]{VoronoiHedcutPaperWithPictures-img/VoronoiHedcutPaperWithPictures-img023.jpg}

\includegraphics[width=1.9134in,height=0.5984in]{VoronoiHedcutPaperWithPictures-img/VoronoiHedcutPaperWithPictures-img025.png}



\bigskip

\ \ \ \ \ In the landscape, I think that Hedcut and Voronoi are better matched. \ Though I still think that Voronoi
produces a better landscape, Hedcut did a decent job of capturing the overall photo.

Here is Voronoi with 40,000 stipples for Painting:


\includegraphics[width=0.8in,height=1.1953in]{VoronoiHedcutPaperWithPictures-img/VoronoiHedcutPaperWithPictures-img026.png}

\includegraphics[width=0.8071in,height=1.2043in]{VoronoiHedcutPaperWithPictures-img/VoronoiHedcutPaperWithPictures-img027.png}


Here is Hedcut with 40,000 stipples for Painting:


\includegraphics[width=0.8in,height=1.1953in]{VoronoiHedcutPaperWithPictures-img/VoronoiHedcutPaperWithPictures-img026.png}

\includegraphics[width=0.8016in,height=1.1957in]{VoronoiHedcutPaperWithPictures-img/VoronoiHedcutPaperWithPictures-img028.png}



\bigskip

\ \ \ \ This I found very surprising. \ Hedcut did a much better job on this particular painting than Voronoi did. \ I
believe that Hedcut performed better due to the low resolution on this image. \ Though you can tell that Voronoi does a
much better job on the edges of the photo than Hedcut. \ If you look at Mona Lisa's hands you can tell that in Hedcut
it is much more blurry than the one in Voronoi.

Here is Voronoi with 40,000 stipples for Photo:


\includegraphics[width=2.5189in,height=1.6752in]{VoronoiHedcutPaperWithPictures-img/VoronoiHedcutPaperWithPictures-img029.jpg}

\includegraphics[width=2.528in,height=1.6827in]{VoronoiHedcutPaperWithPictures-img/VoronoiHedcutPaperWithPictures-img030.png}


Here is Hedcut with 40,000 stipples for Photo:


\includegraphics[width=2.5189in,height=1.6752in]{VoronoiHedcutPaperWithPictures-img/VoronoiHedcutPaperWithPictures-img029.jpg}

\includegraphics[width=2.5154in,height=1.6744in]{VoronoiHedcutPaperWithPictures-img/VoronoiHedcutPaperWithPictures-img031.png}


\ \ \ In this photo, Voronoi does a better job of capturing the baby's face edges than Hedcut does. \ Though I noticed
that both algorithms struggled with the brightness of this photo.

\ \ \ In general, the stippling outputs are very similar to the ones created by artists of the Wall Street Journal,
though I did notice that the artists put heavier accents on the lines of the face than our generated output does. I
wasn't able to examine the differences closely since I would have had to buy a subscription to the Wall Street Journal
in order to read the article. \ I only had access to the video that was linked. \ 


\bigskip

\section[Improve Hedcuter code]{Improve Hedcuter code}

\bigskip

\ \ \ \ I chose to improve Hedcuter code in two ways. \ One was to improve how the color is represented throughout the
image and the second was to make this computation multi-threaded. \ 

\ \ \ \ \ The way I decided to improve the color was to take the original image and split it up into its 3 color
channels; red, green, and blue. By separating the color channels it will make sure that we do not run into the
pigeon-hole property. \ \ That is that we have too many stipples congested in one area which will cause Hedcut to
remove the site altogether (since the Voronoi cell will have zero coverage area). \ This will allow us to represent the
image better since we will not be removing as many sites if we were to do them all at once. 

\ \ \ \ I augmented the code that you can pass the command argument \textbf{{}-channel\_image} which will perform this
enhancement. \ The following is the code snippet that I originally wrote my code (without it being multi-threaded):

\begin{flushleft}
\tablefirsthead{}
\tablehead{}
\tabletail{}
\tablelasttail{}
\begin{supertabular}{|m{6.41436in}|}
\hline
if(channel\_image)

\{

\ \ \ \ \ cv::Mat bgr[3];

\ \ \ \ \ cv::split(input\_image, bgr);

\ \ \ \ //sample n points

\ \ \ \ sample\_initial\_points(bgr[2], n, pts);

\ \ \ \ cvt.compute\_weighted\_cvt(bgr[2], pts);

\ \ \ \ create\_disks(input\_image, cvt);

\ \ \ std::cout {\textless}{\textless} {\textquotedbl}Finished red channel{\textquotedbl} {\textless}{\textless}
{\textquotedbl}{\textbackslash}n{\textquotedbl};

~

~

\ \ \ //sample n points

\ \ \ \ sample\_initial\_points(bgr[0], n, pts);

\ \ \ \ cvt.compute\_weighted\_cvt(bgr[0], pts);

\ \ \ \ create\_disks(input\_image, cvt);

\ \ \ \ std::cout {\textless}{\textless} {\textquotedbl}Finished blue channel{\textquotedbl} {\textless}{\textless}
{\textquotedbl}{\textbackslash}n{\textquotedbl};

~

\ \ \ //sample n points

\ \ \ sample\_initial\_points(bgr[1], n, pts);

\ \ \ cvt.compute\_weighted\_cvt(bgr[1], pts);

\ \ \ create\_disks(input\_image, cvt);

\ \ \ std::cout {\textless}{\textless} {\textquotedbl}Finished green channel{\textquotedbl} {\textless}{\textless}
{\textquotedbl}{\textbackslash}n{\textquotedbl};

\}

~
\\\hline
\end{supertabular}
\end{flushleft}

\bigskip

\ \ \ \ \ I also had to alter the code in Hedcut::create\_disks because originally it would clear out the disks every
time that method was called. \ I had to make sure that the disks were not cleared since I am calling this method on
three separate occasions. \ This uses OpenCV to split the image to each of the Red, Blue, Green channels then calls the
compute\_weighted\_cvt with that matrix. \ It is also important to note that if someone passes in an argument of 8,000
stipples and uses the channel\_image flag that there will be approximately 24,000 stipples in the final image. \ This
is because each channel of the image is contributing approximately 8,000 stipples to the final drawing. \ 

Below is the original image that I will run Hedcut and then the improved Hedcut on:


\includegraphics[width=2.6937in,height=2.1602in]{VoronoiHedcutPaperWithPictures-img/VoronoiHedcutPaperWithPictures-img032.png}


\ Below represents an example of where we run into the problem of sites being removed in the unmodified Hedcut (stipples
150,000) :

\ 
\includegraphics[width=2.8854in,height=2.3146in]{VoronoiHedcutPaperWithPictures-img/VoronoiHedcutPaperWithPictures-img033.png}



\bigskip

Below represents the improved Hedcut using the \textbf{channel\_image} flag with 50,000 stipples (which is tripled to
approximately 150,000 stipples):


\includegraphics[width=3.1445in,height=2.522in]{VoronoiHedcutPaperWithPictures-img/VoronoiHedcutPaperWithPictures-img034.png}



\bigskip


\bigskip

\ \ As you can see, the grass overall is better represented in the \textbf{channel\_image} version than the unimproved
Hedcut version. \ This is because there are less collisions and that less cells are being removed due to the limit of
the resolution of the original image. \ By separating the channels out, you can better represent the image without
having to change the resolution of the image. \ If I had more time I would have improved how the stipples that are
drawn on top of each other to have a bit of transparency. \ Since if two stipples of two different colors channels are
exactly on top of each other, one is blocked by the other. \ Instead I chose to improve this by making it
multi-threaded to speed up the process. \ This did run significantly slower than Hedcut. \ 


\bigskip

\ \ \ \ I hadn't realized how difficult making the program multi-threaded could be since I thought it would be straight
forward. \ My plan was to make each channel calculate the sample points, calculate weighted cvt, and create the disks
on their own thread. \ The first problem I ran into was trying to include {\textless}pthread{\textgreater} as a
dependency. \ I then learned that to add this dependency I had to alter the CMakeList.txt to include the following:

\begin{flushleft}
\tablefirsthead{}
\tablehead{}
\tabletail{}
\tablelasttail{}
\begin{supertabular}{|m{6.41436in}|}
\hline
\#\#\#\#\#\#\#\#\#\# set compiler flags \#\#\#\#\#\#\#\#\#\#

if(NOT WIN32)

\ \ set(CMAKE\_CXX\_FLAGS {\textquotedbl}\$\{CMAKE\_CXX\_FLAGS\} -std=c++11 -pthread{\textquotedbl})

\ \ set(CMAKE\_CXX\_FLAGS {\textquotedbl}\$\{CMAKE\_CXX\_FLAGS\} -Wno-deprecated-declarations{\textquotedbl})

endif()

~

if(WIN32)

\ \ set(CMAKE\_CXX\_FLAGS {\textquotedbl}\$\{CMAKE\_CXX\_FLAGS\} -std=c++11 -pthread /D
\_CRT\_SECURE\_NO\_WARNINGS{\textquotedbl})

endif()

~
\\\hline
\end{supertabular}
\end{flushleft}

\bigskip


\bigskip


\bigskip

\ \ \ \ \ Finally I no longer had a compiler error when I created an std::thread object. \ I ran into another issue
which was when I was creating the thread it said that the method needed to be static. \ I combed through stack overflow
to figure out how to call a non-static method when creating a thread(since there were many instance variables that
these methods had to access). \ \ I had attempted the following:


\bigskip

\begin{flushleft}
\tablefirsthead{}
\tablehead{}
\tabletail{}
\tablelasttail{}
\begin{supertabular}{|m{6.41436in}|}
\hline
for(int i = 0; i {\textless} 3; ++i)

\{

\ \ //where compute\_channel\_image called the various methods

\ \ //sample\_inital\_points, compute\_weighted\_cvt, and create\_disks

\ \ //using the input image given, the bgr[i] matrix that represents a

\ \ //certain channel of the image and n is the number of stippler points

std::thread(\&Hedcut::compute\_channel\_image, this, bgr[i], n, input\_image);

~

\}\\\hline
\end{supertabular}
\end{flushleft}

\bigskip

\ \ \ I learned that you need to pass in \textit{this} as a parameter to allow the method to be a non-static method when
creating the thread. \ But I then ran into an entirely different problem which was that it wouldn't let me pass in
multiple parameters to my \textbf{compute\_channel\_image} method defined in this class when creating a thread. \ This
led me to create a struct in the header file defined as follows:

\begin{flushleft}
\tablefirsthead{}
\tablehead{}
\tabletail{}
\tablelasttail{}
\begin{supertabular}{|m{6.41436in}|}
\hline
struct ChannelParams

\{

\ \ cv::Mat input\_image; \ //the original input image

\ \ int n; \ \ \ \ \ \ \ \ \ \ \ \ \ \ \ //number of stipple points

\ \ cv::Mat channel\_image;//the matrix of the image channel

\ \ int color\_id; //unnecessary, was used for debuging

~

\};\\\hline
\end{supertabular}
\end{flushleft}

\bigskip

\ \ \ By creating this struct, it allowed me to pass a single argument that contained all the values I needed to run the
thread:

\begin{flushleft}
\tablefirsthead{}
\tablehead{}
\tabletail{}
\tablelasttail{}
\begin{supertabular}{|m{6.41436in}|}
\hline
~

for(int i = 0; i {\textless} 3; ++i)

\{

\ \ //where compute\_channel\_image called the various methods

\ \ //sample\_inital\_points, compute\_weighted\_cvt, and create\_disks

\ \ //using the input image given, the bgr[i] matrix that represents a

\ \ //certain channel of the image and n is the number of stippler points

\ \ ChannelParams params;

\ \ params.n = n;

\ \ params.input\_image = input\_image;

\ \ params.channel\_image = bgr[i];

\ \ params.color\_id = i;

std::thread(\&Hedcut::compute\_channel\_image, this,params);

~

\}

~

~
\\\hline
\end{supertabular}
\end{flushleft}

\bigskip


\bigskip

\ \ \ Finally, my code compiles and I thought I was home free! Unfortunately, I ran into race conditions because I had
not made my code wait for the threads to finish. \ So, I had to add a vector containing my threads so that I can join
them and make them wait before I allowed my code to proceed:

\begin{flushleft}
\tablefirsthead{}
\tablehead{}
\tabletail{}
\tablelasttail{}
\begin{supertabular}{|m{6.41436in}|}
\hline
std::vector{\textless}std::thread{\textgreater} threads;

for(int i = 0; i {\textless} 3; ++i)

\{

\ \ //where compute\_channel\_image called the various methods

\ \ //sample\_inital\_points, compute\_weighted\_cvt, and create\_disks

\ \ //using the input image given, the bgr[i] matrix that represents a

\ \ //certain channel of the image and n is the number of stippler points

\ \ ChannelParams params;

\ \ params.n = n;

\ \ params.input\_image = input\_image;

\ \ params.channel\_image = bgr[i];

\ \ params.color\_id = i;

threads.push\_back(std::thread(\&Hedcut::compute\_channel\_image, this,params));

~

\}

~

for(auto\& thread : threads)

\{

\ \ \ \ thread.join(); //this ensures the code doesn't proceed before we are \ \ \ \ \ \ \ \ finished with all the
channels

\}

~
\\\hline
\end{supertabular}
\end{flushleft}

\bigskip

\ \ \ Then I ran into SIGSEGV (address boundary error). \ I quickly realized that I needed to add locks to my code which
meant I needed to include {\textless}mutex{\textgreater}. \ The variable that I wasn't locking and unlocking properly
was the disks variable. \ Since multiple threads could be adding to the disks property simultaneously, this can cause
memory issues. \ I added the following to the create\_disks method:

\begin{flushleft}
\tablefirsthead{}
\tablehead{}
\tabletail{}
\tablelasttail{}
\begin{supertabular}{|m{6.41436in}|}
\hline
this-{\textgreater} lock.lock();

this-{\textgreater} disks.push\_back(disk);

this-{\textgreater} lock.unlock();

~
\\\hline
\end{supertabular}
\end{flushleft}

\bigskip

\ \ \ \ Viola! The color improvement not only fixes the pigeon-hole property but also can run faster since it is using
multiple CPU's. \ I thought for sure that this would run faster than normal Hedcut but unfortunately it did not. \ I
made several attempts to add mutli-threading to wcvt.cpp vor method but without much success. \ I tried to make the
check neighbors double for loop to be multi-threaded. \ My plan was to empty the open list and spawn a thread for each
cell that was in that list and calculate the neighbors distance on its own thread. \ I couldn't figure out how to not
only properly lock the open list (when I needed to add more to it). \ I tried to make the threads that were spawned
that needed to add to the open list to spawn a yet another thread (if it found another candidate) but ran into more
confusing and crazy issues. \ I spent about two days trying to make calculating the neighbors faster but to no avail. 

\begin{flushleft}
\tablefirsthead{}
\tablehead{}
\tabletail{}
\tablelasttail{}
\begin{supertabular}{|m{6.41436in}|}
\hline
//propagate

\ \ while (open.empty() == false)

\ \ \{

\ \ \ \ dist\_lock.lock();

\ \ \ \ std::pop\_heap(open.begin(), open.end(), compareCell);

\ \ \ \ auto cell = open.back();

\ \ \ \ auto\& cpos = cell.second;

\ \ \ \ open.pop\_back();

~

\ \ \ \ //check if the distance from this cell is already updated

\ \ \ \ if (cell.first {\textgreater} dist.at{\textless}float{\textgreater}(cpos.x, cpos.y)) 

\ \ \ \ \{

\ \ \ \ \ \ dist\_lock.unlock();

\ \ \ \ \ \ continue;

\ \ \ \ \}

\ \ \ \ if (visited.at{\textless}uchar{\textgreater}(cpos.x, cpos.y) {\textgreater} 0) 

\ \ \ \ \{

~

\ \ \ \ \ \ dist\_lock.unlock();

\ \ \ \ \ \ continue; 

\ \ \ \ \}//visited

~

\ \ \ \ visited.at{\textless}uchar{\textgreater}(cpos.x, cpos.y) = 1;

~

\ \ \ \ dist\_lock.unlock();

~

\ \ \ \ CheckNeighborsParam param;

\ \ \ \ param.root = \&root;

\ \ \ \ param.resizedImg = resizedImg;

\ \ \ \ param.dist = \&dist;

\ \ \ \ param.res = res;

\ \ \ \ param.open = \&open;

\ \ \ \ param.cpos = cpos;

\ \ \ \ param.dist\_lock = \&dist\_lock;

\ \ \ \ param.other\_lock = \&other\_lock;

\ \ \ \ param.visited = \&visited;

\ \ \ \ threads.push\_back(std::thread(\&CVT::check\_neighbors, this, param));\ \ \ \ 

~

\ \ \}//end while

~

\ \ if(!threads.empty())

\ \ \{

\ \ \ \ for(auto\& mythread : threads)

\ \ \ \ \{

\ \ \ \  \ \ \ mythread.join();

\ \ \ \ \}

\ \ \}

~
\\\hline
\end{supertabular}
\end{flushleft}

\bigskip

The following snippet is the method check\_neighbors:

\begin{flushleft}
\tablefirsthead{}
\tablehead{}
\tabletail{}
\tablelasttail{}
\begin{supertabular}{|m{6.41436in}|}
\hline
void CVT::check\_neighbors(CheckNeighborsParam param)

\{

\ \ int x = param.cpos.x + param.dx;

\ \ if (x {\textless} 0 {\textbar}{\textbar} x {\textgreater}= param.res.height) return;

\ \ for (int dy = -1; dy {\textless}= 1; dy++) //y is column

\ \ \{

\ \ \ \ if (param.dx == 0 \&\& dy == 0) continue; //itself...

~

\ \ \ \ int y = param.cpos.y + dy;

\ \ \ \ if (y {\textless} 0 {\textbar}{\textbar} y {\textgreater}= param.res.width) continue;

\ \ \ \ \ \ \ \ cv::Point pt\_tmp(x, y);

~

\ \ \ \ \ \ \ \ param.lock-{\textgreater}lock();

~

\ \ \ \ float newd = param.dist-{\textgreater}at{\textless}float{\textgreater}(param.cpos.x, param.cpos.y) +
color2dist(param.resizedImg, pt\_tmp);

\ \ \ \ float oldd = param.dist-{\textgreater}at{\textless}float{\textgreater}(x, y);

~

\ \ \ \ if (newd {\textless} oldd)

\ \ \ \ \{

~

\ \ \ \ \ \ param.dist-{\textgreater}at{\textless}float{\textgreater}(x, y)=newd;

\ \ \ \ \ \ param.root-{\textgreater}at{\textless}ushort{\textgreater}(x, y) =
param.root-{\textgreater}at{\textless}ushort{\textgreater}(param.cpos.x, param.cpos.y);

\ \ \ \ \ \ param.open-{\textgreater}push\_back(std::make\_pair(newd, cv::Point(x, y)));

\ \ \ \ \}

\ \ \ \ param.lock-{\textgreater}unlock();

\ \ \}//end for dy

\}\\\hline
\end{supertabular}
\end{flushleft}

\bigskip

\ \ \ \ \ \ I kept running into the SIGSEGV (address boundary error), which means I was not locking up a resource
properly and it was being accessed by two separate threads simultaneously. \ I spent a lot of time trying to figure out
which property was being mishandled and added locks almost everywhere. \ I was unable to track the problem down. 

\ \ I also attempted to make only a portion of it multi-threaded. \ In the double for loop, I tried to spawn threads for
each of the dx's to see if that would speed up the process:

\begin{flushleft}
\tablefirsthead{}
\tablehead{}
\tabletail{}
\tablelasttail{}
\begin{supertabular}{|m{6.41436in}|}
\hline
//propagate

\ \ while (open.empty() == false)

\ \ \{

\ \ \ \ //

\ \ \ \ std::pop\_heap(open.begin(), open.end(), compareCell);

\ \ \ \ auto cell = open.back();

\ \ \ \ auto\& cpos = cell.second;

\ \ \ \ open.pop\_back();

~

\ \ \ \ //check if the distance from this cell is already updated

\ \ \ \ if (cell.first {\textgreater} dist.at{\textless}float{\textgreater}(cpos.x, cpos.y)) continue;

\ \ \ \ if (visited.at{\textless}uchar{\textgreater}(cpos.x, cpos.y) {\textgreater} 0) continue; //visited

\ \ \ \ visited.at{\textless}uchar{\textgreater}(cpos.x, cpos.y) = 1;

~

\ \ \ \ std::mutex lock;

\ \ \ \ //check the neighbors

\ \ \ \ std::vector{\textless}std::thread{\textgreater} threads;

\ \ \ \ for (int dx =-1; dx {\textless}= 1; dx++) //x is row

\ \ \ \ \{

\ \ \ \ \ \ CheckNeighborsParam param;

\ \ \ \ \ \ param.root = \&root;

\ \ \ \ \ \ param.resizedImg = resizedImg;

\ \ \ \ \ \ param.dist = \&dist;

\ \ \ \ \ \ param.res = res;

\ \ \ \ \ \ param.open = \&open;

\ \ \ \ \ \ param.cpos = cpos;

\ \ \ \ \ \ param.dx = dx;

\ \ \ \ \ \ param.lock = \&lock;

\ \ \ \ \ \ threads.push\_back(std::thread(\&CVT::check\_neighbors, this, param));//end thread\ \ \ \ \ \ 

\ \ \ \ \}//end for dx

~

\ \ \ \ \ \ \ \ for(auto\& thread : threads)

\ \ \ \ \ \ \ \ \{

\ \ \ \ \ \ \ \ \ \ thread.join();

\ \ \ \ \ \ \ \ \}

~

\ \ \ \ std::push\_heap(open.begin(), open.end(), compareCell);

\ \ \}//end while

~
\\\hline
\end{supertabular}
\end{flushleft}

\bigskip

\ \ I did get this to work, but there was no speed improvement. \ I had to eventually give up. \ I wanted to show I
could improve the speed of Hedcut but I was only able to successfully improve the speed of my channel\_image
implementation. \ \ In the end, I really improved my c++ abilities and learned a lot more about how to multi-thread
programs in c++ which was something I had not ever done before. 
\end{document}
